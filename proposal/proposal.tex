%%%%%%%%%%%%%%%%%%%%%%%%%%%%%%%%%%%%%%%%%
% Journal Article
% LaTeX Template
% Version 1.3 (9/9/13)
%
% This template has been downloaded from:
% http://www.LaTeXTemplates.com
%
% Original author:
% Frits Wenneker (http://www.howtotex.com)
%
% License:
% CC BY-NC-SA 3.0 (http://creativecommons.org/licenses/by-nc-sa/3.0/)
%
%%%%%%%%%%%%%%%%%%%%%%%%%%%%%%%%%%%%%%%%%

%----------------------------------------------------------------------------------------
% PACKAGES AND OTHER DOCUMENT CONFIGURATIONS
%----------------------------------------------------------------------------------------

\documentclass[twoside]{article}

\usepackage{lipsum} % Package to generate dummy text throughout this template

\usepackage[sc]{mathpazo} % Use the Palatino font
\usepackage[T1]{fontenc} % Use 8-bit encoding that has 256 glyphs
\linespread{1.05} % Line spacing - Palatino needs more space between lines
\usepackage{microtype} % Slightly tweak font spacing for aesthetics

\usepackage[hmarginratio=1:1,top=32mm,columnsep=20pt]{geometry} % Document margins
\usepackage{multicol} % Used for the two-column layout of the document
\usepackage[hang, small,labelfont=bf,up,textfont=it,up]{caption} % Custom captions under/above floats in tables or figures
\usepackage{booktabs} % Horizontal rules in tables
\usepackage{float} % Required for tables and figures in the multi-column environment - they need to be placed in specific locations with the [H] (e.g. \begin{table}[H])
\usepackage{hyperref} % For hyperlinks in the PDF

\usepackage{lettrine} % The lettrine is the first enlarged letter at the beginning of the text
\usepackage{paralist} % Used for the compactitem environment which makes bullet points with less space between them

\usepackage{abstract} % Allows abstract customization
\renewcommand{\abstractnamefont}{\normalfont\bfseries} % Set the "Abstract" text to bold
\renewcommand{\abstracttextfont}{\normalfont\small\itshape} % Set the abstract itself to small italic text

\usepackage{titlesec} % Allows customization of titles
\renewcommand\thesection{\Roman{section}} % Roman numerals for the sections
\renewcommand\thesubsection{\Roman{subsection}} % Roman numerals for subsections
\titleformat{\section}[block]{\large\scshape\centering}{\thesection.}{1em}{} % Change the look of the section titles
\titleformat{\subsection}[block]{\large}{\thesubsection.}{1em}{} % Change the look of the section titles

\usepackage{fancyhdr} % Headers and footers
\pagestyle{fancy} % All pages have headers and footers
\fancyhead{} % Blank out the default header
\fancyfoot{} % Blank out the default footer
\fancyhead[C]{6.s062 Final Project Proposal $\bullet$ March 2016} % Custom header text
\fancyfoot[RO,LE]{\thepage} % Custom footer text

%----------------------------------------------------------------------------------------
% TITLE SECTION
%----------------------------------------------------------------------------------------

\title{\vspace{-15mm}\fontsize{24pt}{10pt}\selectfont\textbf{Secure BLE-Based Authentication using Sensor Augmentation}}

\author{
\large
\textsc{Austin Freel, Andrew Titus}\\[2mm] % Your name
\normalsize Massachusetts Institute of Technology \\ % Your institution
\normalsize [\href{mailto:afreel@mit.edu}{afreel}, \href{mailto:atitus@mit.edu}{atitus}]@mit.edu % Your email address
\vspace{-5mm}
}
\date{}

%----------------------------------------------------------------------------------------

\begin{document}

\maketitle % Insert title

\thispagestyle{fancy} % All pages have headers and footers

%----------------------------------------------------------------------------------------
% ARTICLE CONTENTS
%----------------------------------------------------------------------------------------

\begin{multicols}{2} % Two-column layout throughout the main article text

\section{Introduction}

\lettrine[nindent=0em,lines=3]{M}any multi-factor authentication schemes using external
hardware exist, but few utilize the existing Bluetooth Low Energy (BLE) and sensor hardware
on people's mobile devices. This lack of ubiquity can lead to significant ``friction'' in
users - that is, users experience a level of frustration every time they need to use a
separate device that can often lead to users opting for less secure but easier to use methods.
Fortunately, most mobile devices manufactured in the past few years contain hardware that
enables BLE and various sensors to be utilized for wireless communications and data collection.
This leads us to propose that a ``low-friction'' system can be built for existing mobile
hardware that guarantees the same level of security for multi-factor authentication.

%------------------------------------------------

\section{Methodology}

Many Bluetooth-enabled peripherals (including mobile devices) have the ability to support
encrypted communication. We will build our system on iOS, as the operating system requires
that all remote devices connected to it support encrypted pairing connections
(\href{https://support.apple.com/en-us/HT204387}{see here}). We will start out by developing
a pair of iOS 9 applications, one of which will function as a normal user and one of which will function
as the BLE peripheral to which to send encrypted credentials. This will enable us to test
the pairing and authentication processes without using external hardware for peripherals
during testing. Then, we plan to replace this BLE peripheral iPhone with an external
hardware device, such as an Arduino-based device or a custom-built Bluetooth peripheral (perhaps
a 6.115 project by Andrew).

The setup of the pairing connection will be done by standard Apple guidelines for connecting
Bluetooth devices. Authentication methods, however, are much more flexible and wide-ranging.
We will initially start with a simple one-factor authentication system, such as a username/password
combination or certificate-based system. Then, we will move into multi-factor authentication
schemes which are augmented by internal sensor data, such as gestures. We also hope to make the
system more robust to spurious authentications by using sensor data as well. For example, in the
use case of the BLE peripheral being a door lock, this would prevent users from unlocking a door
if they are simply walking by it.

%------------------------------------------------

\section{Plan and Schedule}

Assuming a due date of May 9th, we will have exactly two months to complete this project. We present
here a list of deliverables, with our current plan and due date estimates:
\begin{compactitem}
    \item{\textit{Simple communication between user and peripheral iOS applications}: We will begin design work and begin constructing the basic wireframes of the test applications by \textbf{March 18th}, the start of Spring Break. We will then build the applications and establish this simple communication by \textbf{April 1st}.}
    \item{\textit{One-factor authentication between user and peripheral iOS applications}: This will likely not need much additional work to the simple communication applications, so we have the goal of establishing a one-factor authentication scheme (likely with simple dummy passwords) by \textbf{April 8th}.}
    \item{\textit{Multi-factor authentication between user and peripheral iOS applications}: This is a somewhat heftier goal that will require a great deal of research, design decisions regarding what factors to use, and testing, so we are setting the due date for this to be \textbf{April 29th}.}
    \item{\textit{Optimization for spurious authentications}: This is a somewhat unbounded goal, so we will aim to have some degree of completion on this by the end of the project on \textbf{May 9th}.}
\end{compactitem}

As the 6.115 Final Project Proposal is due April 5th, Andrew will have decided by this point what type
of project he will be pursuing for this class. If it is possible to build a BLE peripheral for this project,
we will focus concurrently on integrating this hardware using the same steps (simple communication,
one-factor, multi-factor, optimization). Otherwise, we will begin similar work and research on an Arduino-based
alternative. Our main deliverable is a proof-of-concept, so if the external hardware is not working as well
as we hope, we will focus on making the phone-phone BLE multi-factor authentication system work.

%------------------------------------------------

\section{Resources and Requirements}

In order to accomplish this work, we will need xCode 7 and at least two iOS 9+ iPhones, which we
currently own. Additionally, we will need at least one external peripheral as mentioned,
which will consist of either Arduino-based devices similar to the Anthill or a custom-built
BLE device that Andrew will construct in 6.115 (Microprocessor Project Laboratory). We may
also need laboratory equipment or specialized computers to be able to program these external
peripherals. If Arduino-based devices are chosen, this should likely not pose much of an issue,
but if the device is custom-built, Andrew will seek the proper permissions from the 6.115 laboratory
staff to build such a device in accordance with our project.

\end{multicols}

\end{document}
